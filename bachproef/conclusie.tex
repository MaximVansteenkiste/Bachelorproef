\chapter{Conclusie}
\label{ch:conclusie}

In de twee vorige hoofdstukken werden de gekozen build tools op de proef gesteld aan de hand van drie open-source projecten en een van nul te beginnen. De verschillende stappen om ze werkende te krijgen werden overlopen, bij de een waren dat er al meer dan de ander. Bij elk van die twee hoofdstukken werd op het einde ook een aparte conclusie getrokken. Dit hoofdstuk dient als algemene conclusie. 

De hoofdvraag van deze proef, of Webpack nog steeds mag gezien worden als de koning der Javascript build tools, kan beantwoord worden met een duidelijke ‘nee’. Zelf binnen zijn eigen categorie van module bundlers, blijkt Parcel, in de meeste gevallen, een betere optie te zijn. Vite bewees dat de ongebundelde build tools de toekomst zijn. Module bundlers zijn hierdoor echter nog niet voorbijgestreefd. Zelfs wanneer browsers en Node.js ESModules volledig ondersteunen en alle open-source packages er volledig mee geschreven zijn, zullen de voordelen van module bundlers hun plaats in de Javascript wereld garanderen. Voor de nabije toekomst zullen module bundlers en ongebundelde build tools hand in hand moeten samenwerken om het beste van beide werelden samen te brengen. Vite lijkt dit het best te begrijpen. Het is dus wel duidelijk dat Vite de beste optie is voor zowel een nieuwe webapplicatie te ontwikkelen, als een bestaand een nieuwe, snellere en modernere motor te geven. 

