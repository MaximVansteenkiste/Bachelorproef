

\makeglossaries

\newglossaryentry{HTML}
{
    name=HTML,
    description={Hyper Text Markup Language is een opmaaktaal voor de specificatie van documenten, voornamelijk bedoeld voor het web}
}

\newglossaryentry{CSS}
{
    name=CSS,
    description={Cascading Style Sheets bieden een mogelijkheid om de vormgeving van webpagina's los te koppelen van hun feitelijke inhoud}
}

\newglossaryentry{Javascript}
{
    name=Javascript,
    description={Een veelgebruikte scripttaal om webpagina's interactief te maken en webapplicaties te ontwikkelen}
}

\newglossaryentry{open-source}
{
    name=open-source,
    description={Het openbaar maken van de broncode van een applicatie zodat anderen het (gratis) kunnen gebruiken}
}

\newglossaryentry{packages}
{
    name=packages,
    description={Software gebundeld tot een pakket die functies bevat die een applicatie kan gebruiken}
}

\newglossaryentry{package-manager}
{
    name=package-manager,
    description={Een tool die het downloaden van packages toelaat}
}

\newglossaryentry{SASS}
{
    name=SASS,
    description={Een superset of uitbreiding bovenop CSS}
}

\newglossaryentry{web frameworks}
{
    name=web frameworks,
    description={Web frameworks bieden een standaardmethode om een website te ontwikkelen en te publiceren}
}

\newglossaryentry{JSON}
{
    name=JSON,
    description={JavaScript Object Notation, is een gestandaardiseerd gegevensformaat}
}