%%=============================================================================
%% Samenvatting
%%=============================================================================

% TODO: De "abstract" of samenvatting is een kernachtige (~ 1 blz. voor een
% thesis) synthese van het document.
%
% Deze aspecten moeten zeker aan bod komen:
% - Context: waarom is dit werk belangrijk?
% - Nood: waarom moest dit onderzocht worden?
% - Taak: wat heb je precies gedaan?
% - Object: wat staat in dit document geschreven?
% - Resultaat: wat was het resultaat?
% - Conclusie: wat is/zijn de belangrijkste conclusie(s)?
% - Perspectief: blijven er nog vragen open die in de toekomst nog kunnen
%    onderzocht worden? Wat is een mogelijk vervolg voor jouw onderzoek?
%
% LET OP! Een samenvatting is GEEN voorwoord!

%%---------- Nederlandse samenvatting -----------------------------------------
%
% TODO: Als je je bachelorproef in het Engels schrijft, moet je eerst een
% Nederlandse samenvatting invoegen. Haal daarvoor onderstaande code uit
% commentaar.
% Wie zijn bachelorproef in het Nederlands schrijft, kan dit negeren, de inhoud
% wordt niet in het document ingevoegd.

\IfLanguageName{english}{%
\selectlanguage{dutch}
\chapter*{Samenvatting}
\lipsum[1-4]
\selectlanguage{english}
}{}

%%---------- Samenvatting -----------------------------------------------------
% De samenvatting in de hoofdtaal van het document

\chapter*{\IfLanguageName{dutch}{Samenvatting}{Abstract}}
Een module bundler of build tool is een essentieel onderdeel in het maken van moderne webapplicaties. Al jaren wordt Webpack gezien als de beste keuze hiervoor. Aangezien er genoeg andere opties zijn, vele met een nieuwe aanpak, is het wel eens tijd om die bewering nog eens na te gaan. In een literatuurstudie wordt de geschiedenis en werking van een module bundler geschetst. Daarna, aan de hand van een vergelijkende studie, worden drie populaire alternatieven tegenover Webpack geplaatst in rechtstreeks duel. Eerst wordt een nieuw project opgezet met de respectievelijke kandidaten. Daarna trachten we drie open-source projecten, elk met zijn eigen moeilijkheden, om te vormen van Webpack naar een tegenkandidaat. Doorheen dit gedocumenteerd proces, wordt er meer uitleg gegeven over de verschillende technologieën die aan bod komen. Tot slot wordt in de conclusie, rekening houdend met verschillende ob- en subjectieve factoren, de titel van deze proef beantwoord. 
