%===============================================================================
% LaTeX sjabloon voor de bachelorproef toegepaste informatica aan HOGENT
% Meer info op https://github.com/HoGentTIN/bachproef-latex-sjabloon
%===============================================================================

\documentclass{bachproef-tin}

\usepackage{hogent-thesis-titlepage} % Titelpagina conform aan HOGENT huisstijl
\usepackage{listings}
\usepackage{xcolor}
\usepackage{booktabs}
\usepackage[utf8]{inputenc}
\usepackage[nonumberlist]{glossaries}



\makeglossaries

\newglossaryentry{HTML}
{
    name=HTML,
    description={Hyper Text Markup Language is een opmaaktaal voor de specificatie van documenten, voornamelijk bedoeld voor het web.}
}

\newglossaryentry{CSS}
{
    name=CSS,
    description={Cascading Style Sheets bieden een mogelijkheid om de vormgeving van webpagina's los te koppelen van hun feitelijke inhoud}
}

\newglossaryentry{Javascript}
{
    name=Javascript,
    description={Een veelgebruikte scripttaal om webpagina's interactief te maken en webapplicaties te ontwikkelen}
}

\newglossaryentry{open-source}
{
    name=open-source,
    description={Het openbaar maken van de broncode van een applicatie zodat anderen het (gratis) kunnen gebruiken}
}

\newglossaryentry{packages}
{
    name=packages,
    description={Software gebundeld tot een pakket die functies bevat die een applicatie kan gebruiken}
}

\newglossaryentry{package-manager}
{
    name=packages,
    description={Een tool die het downloaden van packages toelaat}
}

\newglossaryentry{SASS}
{
    name=SASS,
    description={Een superset of uitbreiding bovenop CSS}
}

\definecolor{codegreen}{rgb}{0,0.6,0}
\definecolor{codegray}{rgb}{0.5,0.5,0.5}
\definecolor{codepurple}{rgb}{0.58,0,0.82}

\lstdefinestyle{codeStyle}{  
    belowcaptionskip=1\baselineskip,
    breaklines=true,
    frame=none,
    numbers=none, 
    basicstyle=\footnotesize\ttfamily,
    keywordstyle=\bfseries\color{green!40!black},
    commentstyle=\itshape\color{purple!40!black},
    identifierstyle=\color{blue},
    backgroundcolor=\color{gray!10!white},
}

\lstset{style=codeStyle}

\lstdefinelanguage{JavaScript}{
  keywords={break, case, catch, continue, debugger, default, delete, do, else, false, finally, for, function, if, in, instanceof, new, null, return, switch, this, throw, true, try, typeof, var, void, while, with},
  morecomment=[l]{//},
  morecomment=[s]{/*}{*/},
  morestring=[b]',
  morestring=[b]",
  ndkeywords={class, export, boolean, throw, implements, import, this},
  keywordstyle=\color{blue}\bfseries,
  ndkeywordstyle=\color{darkgray}\bfseries,
  identifierstyle=\color{black},
  commentstyle=\color{purple}\ttfamily,
  stringstyle=\color{red}\ttfamily,
  sensitive=true
}



%%---------- Documenteigenschappen ---------------------------------------------
% TODO: Vul dit aan met je eigen info:

% De titel van het rapport/bachelorproef
\title{Mag Webpack zich nog steeds de koning der Javascript build tools noemen?}

% Je eigen naam
\author{Maxim Vansteenkiste}

% De naam van je promotor (lector van de opleiding)
\promotor{Heidi Roobrouck}

% De naam van je co-promotor. Als je promotor ook je opdrachtgever is en je
% dus ook inhoudelijk begeleidt (en enkel dan!), mag je dit leeg laten.
\copromotor{Cedric Vanhaeverbeke}

% Indien je bachelorproef in opdracht van/in samenwerking met een bedrijf of
% externe organisatie geschreven is, geef je hier de naam. Zoniet laat je dit
% zoals het is.
\instelling{---}

% Academiejaar
\academiejaar{2021-2022}

% Examenperiode
%  - 1e semester = 1e examenperiode => 1
%  - 2e semester = 2e examenperiode => 2
%  - tweede zit  = 3e examenperiode => 3
\examenperiode{1}

%===============================================================================
% Inhoud document
%===============================================================================

\begin{document}

%---------- Taalselectie -------------------------------------------------------
% Als je je bachelorproef in het Engels schrijft, haal dan onderstaande regel
% uit commentaar. Let op: de tekst op de voorkaft blijft in het Nederlands, en
% dat is ook de bedoeling!

% \selectlanguage{english}

%---------- Titelblad ----------------------------------------------------------
\inserttitlepage

%---------- Samenvatting, voorwoord --------------------------------------------
\usechapterimagefalse
%%=============================================================================
%% Voorwoord
%%=============================================================================

\chapter*{\IfLanguageName{dutch}{Woord vooraf}{Preface}}
\label{ch:voorwoord}

%% TODO:
%% Het voorwoord is het enige deel van de bachelorproef waar je vanuit je
%% eigen standpunt (``ik-vorm'') mag schrijven. Je kan hier bv. motiveren
%% waarom jij het onderwerp wil bespreken.
%% Vergeet ook niet te bedanken wie je geholpen/gesteund/... heeft

Deze bachelorproef werd geschreven in functie van het afronden van de opleiding Toegepaste Informatica aan de Hogeschool van Gent. Ook al volg ik de afstudeerrichting mobiele applicaties, waar webapplicaties een deel van zijn, toch is de module bundler nooit aan bod gekomen. Hoewel velen erop vertrouwen, weten weinigen wat ze voor essentiële taken verrichten. Door deze proef hoop ik dat er meer wordt stilgestaan bij het kiezen van een module bundler. 

Ik wil graag mijn promotor, Heidi Roobrouck, bedanken om deze scriptie in goede banen te leiden. Ook mijn co-promotor, Cedric Vanhaeverbeke, verdient een dankwoordje. Ik kon steeds rekenen op zijn uitgebreide feedback. Deze twee personen hebben ervoor gezorgd dat dit onderzoek tot een goed einde is gekomen.

\IfLanguageName{english}{%
\selectlanguage{dutch}
\chapter*{Samenvatting}
\lipsum[1-4]
\selectlanguage{english}
}{}

%%---------- Samenvatting -----------------------------------------------------
% De samenvatting in de hoofdtaal van het document

\chapter*{\IfLanguageName{dutch}{Samenvatting}{Abstract}}
Een Javascript build tool is een essentieel onderdeel bij het maken van moderne webapplicaties. Al jaren wordt Webpack gezien als de beste keuze hiervoor. Aangezien er genoeg andere opties zijn, vele met een nieuwe aanpak, is het wel eens tijd om die bewering te controleren. In een literatuurstudie wordt de geschiedenis en werking van een Javascript build tool geschetst. Daarna, aan de hand van een vergelijkende studie, worden drie populaire alternatieven tegenover Webpack geplaatst in rechtstreeks duel. Eerst wordt een nieuw project opgezet met de respectievelijke kandidaten. Daarna trachten we drie open-source projecten, elk met zijn eigen moeilijkheden, om te vormen van Webpack naar een tegenkandidaat. Doorheen dit gedocumenteerd proces, wordt er meer uitleg gegeven over de verschillende technologieën die aan bod komen. Tot slot wordt in de conclusie, rekening houdend met verschillende ob- en subjectieve factoren, de titel van deze proef beantwoord.




%---------- Inhoudstafel -------------------------------------------------------
\pagestyle{empty} % Geen hoofding
\tableofcontents  % Voeg de inhoudstafel toe
\cleardoublepage  % Zorg dat volgende hoofstuk op een oneven pagina begint
\pagestyle{fancy} % Zet hoofding opnieuw aan

%---------- Lijst figuren, afkortingen, ... ------------------------------------

% Indien gewenst kan je hier een lijst van figuren/tabellen opgeven. Geef in
% dat geval je figuren/tabellen altijd een korte beschrijving:
%
%  \caption[korte beschrijving]{uitgebreide beschrijving}
%
% De korte beschrijving wordt gebruikt voor deze lijst, de uitgebreide staat bij
% de figuur of tabel zelf.

\listoffigures
\listoftables

% Als je een lijst van afkortingen of termen wil toevoegen, dan hoort die
% hier thuis. Gebruik bijvoorbeeld de ``glossaries'' package.
% https://www.overleaf.com/learn/latex/Glossaries

\pagestyle{empty}
\printglossary
\pagestyle{fancy}

%---------- Kern ---------------------------------------------------------------

% De eerste hoofdstukken van een bachelorproef zijn meestal een inleiding op
% het onderwerp, literatuurstudie en verantwoording methodologie.
% Aarzel niet om een meer beschrijvende titel aan deze hoofstukken te geven of
% om bijvoorbeeld de inleiding en/of stand van zaken over meerdere hoofdstukken
% te verspreiden!

%%=============================================================================
%% Inleiding
%%=============================================================================

\chapter{\IfLanguageName{dutch}{Inleiding}{Introduction}}
\label{ch:inleiding}

De wijze waarop een webapplicatie gemaakt wordt, verandert continu. Het lijkt wel of er elke week een nieuw framework uitkomt met een iets andere aanpak dan al degene die hem voorgingen. Eén ding echter blijft al enkele jaren hetzelfde: de nood aan een module bundler is er nog steeds, voor nu toch.

Dit onderzoek tracht de noodzaak, werking en toekomst van de module bundler te schetsen. Dit aan de hand van een paar veel voorkomende implementaties ervan. Het uiteindelijke doel is om te bepalen of de huidige koning van de module bundlers, Webpack, nog steeds het recht heeft om die positie op te eisen.

\section{\IfLanguageName{dutch}{Probleemstelling}{Problem Statement}}
\label{sec:probleemstelling}

Dit onderzoek is in de eerste plaats gericht naar mijn co-promotor. Als webdeveloper komt hij vaak voor de keuze van welke module bundler te gebruiken. In de toekomst kan hij die beslissing dus maken aan de hand van deze vergelijkende studie. Daarnaast kan dit onderzoek ook een meerwaarde bieden voor andere webdevelopers, inclusief mezelf. 

\section{\IfLanguageName{dutch}{Onderzoeksvraag}{Research question}}
\label{sec:onderzoeksvraag}

De hoofdonderzoeksvraag werd al eerder aangehaald: is het nog steeds gewettigd dan Webpack de meest gebruikte module bundler is? 

\section{\IfLanguageName{dutch}{Onderzoeksdoelstelling}{Research objective}}
\label{sec:onderzoeksdoelstelling}

Hoewel de onderzoeksvraag beantwoorden aan de hand van een vergelijkende studie uiteraard het uiteindelijke doel is, hoop ik dat dit werk volgend doel ook verwezenlijkt.

De module bundler is voor velen, mijn co-promotor en mezelf erbij gerekend, iets wat ze gebruiken en nodig hebben maar niet zoveel over nadenken en weten. Vele frameworks om webapplicaties te maken komen met een module bundler ingebouwd. Zo wordt de keuze dus voor je gemaakt. Ideaal voor iemand die zich daar geen zorgen over wil maken maar aangezien de webapplicatie niet werkt zonder, is het de moeite om toch meer te weten erover. Het tweede, meer verborgen doel is dus om de module bundler en zijn werking te ontrafelen. 

\section{\IfLanguageName{dutch}{Opzet van deze bachelorproef}{Structure of this bachelor thesis}}
\label{sec:opzet-bachelorproef}

% Het is gebruikelijk aan het einde van de inleiding een overzicht te
% geven van de opbouw van de rest van de tekst. Deze sectie bevat al een aanzet
% die je kan aanvullen/aanpassen in functie van je eigen tekst.

De rest van deze bachelorproef is als volgt opgebouwd:

In Hoofdstuk~\ref{ch:stand-van-zaken} wordt een overzicht gegeven van de stand van zaken binnen het onderzoeksdomein, op basis van een literatuurstudie.

In Hoofdstuk~\ref{ch:methodologie} wordt de methodologie toegelicht en worden de gebruikte onderzoekstechnieken besproken om een antwoord te kunnen formuleren op de onderzoeksvragen.

% TODO: Vul hier aan voor je eigen hoofstukken, één of twee zinnen per hoofdstuk

In Hoofdstuk~\ref{ch:conclusie}, tenslotte, wordt de conclusie gegeven en een antwoord geformuleerd op de onderzoeksvragen. Daarbij wordt ook een aanzet gegeven voor toekomstig onderzoek binnen dit domein.
\chapter{State of the art}
\label{ch:stand-van-zaken}

% Tip: Begin elk hoofdstuk met een paragraaf inleiding die beschrijft hoe
% dit hoofdstuk past binnen het geheel van de bachelorproef. Geef in het
% bijzonder aan wat de link is met het vorige en volgende hoofdstuk.

% Pas na deze inleidende paragraaf komt de eerste sectiehoofding.

A modern website is created with three main technologies: HTML for content, CSS for styling, and Javascript to make the page interactive. Those three can all be linked, put on a server somewhere and the end-user can load a working website. Even though the large web platforms of today, like Facebook and Youtube, are all based on these technologies, a lot more ingredients are needed to make them work.

If we were to create a static, basic site, the three main technologies of the web are all you need. But it’s a whole different story for a developer that wants to create a more complex site, a reactive site, a site that uses open-source packages or just wants to make his own job a little bit easier. There are thousands of libraries, packages and frameworks that enable web development on a whole new level and easier than ever before. However this does pose a new problem. 

If we were to create a web application with a single Javascript file and no dependencies i.e no other files that are linked to that Javascript file, link it to our index.html and throw in some CSS, our site would work fine. But what if we introduce a second Javascript file and reference it in our other file. What if we want to install and use an open-source package downloaded from a package-manager like NPM? What if we want to use SASS to extend CSS? This all wouldn’t work out of the box. The browser just isn’t able to figure out how to stitch all the different pieces together.
We need something that will bundle all the different Javascript files and dependencies, something that understands and correctly loads the SASS files or any other file for that matter. We need a module bundler. In essence, they take all the different source files in a project and put them into a single output file that the browser understands. 

\section{History}

To understand the rest of this paper, you’ll need to know what a module is. Modular programming breaks a program up into chunks based on functionality and often in separate files. These chunks are called modules. Modules are then linked together to form an application. 

Node.js, a Javascript runtime for computers and servers, has supported modular programming since the beginning using CommonJS. However support for modules on the web has been slow to arrive. The first module bundlers only came around after 2014. To understand why module bundlers exist, we should first know how web applications were built before them and what problem they solved.

\subsection{Script tags}

Javascript can be linked to, or written directly in, the HTML file of a site using script-tags. Knowing this, we could use a different tag and corresponding Javascript file for each use case. Let’s say we have a file with all the code for authenticating a user and another one for general events (button clicks, ...). This is fine when we only have two files that aren’t that big, but introduces network bottlenecks when scaled to a larger application. The same is true if we were to put all our code in one large JS file and link it to the HTML. The global scope would also get polluted with our custom functions. As some, or all, the functions are available on the global scope, this could introduce security risks and name collisions. 

\lstinputlisting[language=HTML]{codeSnippets/scriptTags.html}

\subsection{Immediately invoked function expressions}

An Immediately invoked function expression or IIFE is a function that runs as soon as it’s defined. Because each IIFE declares a local scope, it solves the problem of polluting the global one. The use of IFFEs led to so-called task runners: they concatenate all your project files together. The big drawback of task runners is that when one file is changed, the whole project has to be rebuilt. You are also required to manually define all the dependencies up front. They make it easier to reuse functions and whole scripts but do nothing for the build output. You can still end up with a very large Javascript file that the user has to download. \\ \\ \\

\lstinputlisting[language=Javascript]{codeSnippets/IIFE.js}

\subsection{CommonJS}

Before 2009, Javascript ran only in a browser. Node.js introduced a Javascript runtime that could run on computers and servers. This introduced a new set of challenges. As Javascript wasn’t run in the browser and therefore no HTML script-tags were around, how could those applications load new chunks of code? 

CommonJS introduced the require function in Javascript. With it, everything that an external module exports, can be imported. Reusable code can now be imported from any other Javascript file in a project. It makes implementing dependency management easy to understand.

All this came with a big catch: It worked, and still works, great for Node.js applications but it isn’t an official feature of Javascript and therefore browsers don’t support it. As commonJS doesn’t actually bundle the code, web browsers can’t make sense of the different modules that are imported. Something has to do it for them. 

\lstinputlisting[language=Javascript]{codeSnippets/commonJSExport.js}

\lstinputlisting[language=Javascript]{codeSnippets/commonJSImport.js}

\subsection{ECMAScript Modules}

CommonJS wasn’t an official feature of Javascript. ECMAscript (=Javascript) did introduce its own module system in version 6. ECMAScript Modules or ESM, accomplish the same goals as CommonJS, but with a different syntax. Now modern browsers can make sense of modular applications that only use ESMs. 

\lstinputlisting[language=Javascript]{codeSnippets/esmExport.js}

\lstinputlisting[language=Javascript]{codeSnippets/esmImport.js}

\section{Module bundlers}

Developers are always seeking ways to make their lives easier. They wanted to import whatever type of module or any asset for that matter into their project and ship it in a smaller output file than the source to the end-user. This is why the module bundler was born. 

All module bundlers around today share common concepts. We’ll look at the 2 most important ones. 

\subsubsection{Tree shaking}
Tree shaking is the process of removing dead code. When a module is imported, perhaps only a part of that module is needed. Maybe only one function of that module is used in the project. With tree shaking, the rest of that module that isn’t used, is removed. 

\subsubsection{Code splitting}
Code splitting can be used to divide up the output bundle that is created into smaller files. The partial bundles are then loaded in parallel or when needed. For example: take a website with multiple pages. If the code isn’t split, all the code of all the pages will be contained in a single file and downloaded by the user when the site is requested. In many cases, like a small project, this is fine. That’s why it’s optional. For larger applications however, it could introduce network bottlenecks. If that’s the case, the code can be split into a single or multiple files per page. Then the smaller files are only loaded when the corresponding page is requested. 

\subsection{Bundled development}

In this section, examples of module bundlers that use bundled development will be discussed. Almost every Javascript web bundler is based on the concepts of bundled development. The alternative and more modern approach is unbundled development. The differences and examples will be explained in the next section. 

\subsubsection{Webpack}

Webpack is the most popular module bundler in the world according to NPM downloads. This has much to do with the fact that it's been around the longest. It comes pre-installed in many web frameworks like create-react-app and Next.js. This way, it’s used by many without even knowing it. It comes with an optional built in development server that makes setting up a local development environment easier.

Webpack runs on Node.js. It can do its job without any configuration, however is very configurable if needed. It supports the module types discussed above and more. 
Because Webpack only understands Javascript and JSON files out of the box, Loaders are used to allow processing of other file types and convert them into valid modules.
Using Loaders, other types of modules or even assets like images can be imported and processed by WebPack. On top of that, Webpack can also be extended with plugins. They allow for a wide range of extra functionality like bundle optimization. 

The function of a module bundler has already been discussed. But how does Webpack achieve this? Whenever one file depends on another in a project, Webpack sees this dependency and puts it into something called a dependency graph. 

This graph is built recursively. When it is time to build the application, it uses the graph to piece all the files together into one output file that is then shipped to the browser. Webpack does this both in development as in production. 

\includegraphics{Webpack}

\subsubsection{Rollup and Parcel}
...
\subsection{Unbundled development}

The examples we looked at so far are all module bundlers that use the concepts of bundled development. One very important thing to note is the word development. The differences between both approaches lies only in the development phase. The advantages of unbundled development are only noticeable when the developer or the development team is actually developing the application. When it is time to make a production build that the end-user will be able to use, the following unbundled development build tools still use the same module bundlers discussed in the previous section. So in essence, we can’t call the unbundled development build tools module bundlers, because they don’t bundle anything. They do use module bundlers.

Because most browsers now support ESM, bundling all Javascript modules of a project into one output file isn’t necessary. All non-Javascript modules do have to be translated or built to a valid Javascript module using a built tool (in WebPack this is called a Loader). In bundled development, when the development server is started, the whole project first has to be built and bundled; When a file is changed that file is rebuilt and then the whole project is again rebundled. In unbundled development a file is only built when it’s requested, meaning very fast startup time of the server. When a file is built, it is cached indefinitely. The browser will never have to download a file twice until it changes. When a file does change, only that single file has to be rebuilt. 

\includegraphics[scale=0.6]{Bundledvsunbundled}

All this results in very fast development performance compared to bundled development tools like WebPack. However, as mentioned before, unbundled development tools still use traditional module bundlers to make a production build. It does this because the advantages of Tree-shaking, a single output file and other features of module bundlers still hold up in the production phase: Tree-shaking results in a smaller output file; importing non-Javascript or CommonJS modules into ESMs is not supported in browsers; Not all modules are written with ESM; and the list goes on. We’ll now look at two examples of unbundled development tools.

\subsubsection{Snowpack}
...
\subsubsection{Vite}
...

%%=============================================================================
%% Methodologie
%%=============================================================================

\chapter{\IfLanguageName{dutch}{Methodologie}{Methodology}}
\label{ch:methodologie}

In de stand van zaken werden de verschillende te vergelijken module bundlers toegelicht. Deze zullen in dit hoofdstuk naast elkaar gelegd en gequoteerd worden aan de hand van verscheidene factoren. 
Eerst zal gekeken worden hoe ze verschillen bij het opzetten van een nieuw project. Daarna trachten we een bestaand project dat met Webpack opgezet is, om te vormen naar een van de andere opties. In het volgend hoofdstuk volgt de conclusie.

Eerst en vooral is er nood aan bestaande projecten om de build tools te kunnen testen. Gelukkig zijn er online genoeg open-source voorbeelden hiervan te vinden. Drie projecten werden gekozen aan de hand van hun grootte en technologieën die ze gebruiken. Eén eigenschap hebben ze allemaal gemeen: het zijn Javascript projecten die React als UI library gebruiken. De reden dat geen projecten gekozen zijn die andere UI libraries zoals Vue gebruiken, is omdat React veel meer gebruikt wordt, ook door de co-promotor. Desondanks zijn de gekozen projecten ook representatief voor de andere UI libraries omdat ze uiteindelijk allemaal Javascript zijn. 

Volgende factoren worden gebruikt om een eenduidige vergelijking te maken. Daarnaast zullen meer subjectieve factoren, gelijk de gemakkelijkheid om het werkende te krijgen, aan bod komen.
\begin{itemize}
    \item Output bundle grootte
    \item Output bundle snelheid
    \item Snelheid bij wijziging
    \item Opstartsnelheid development server
\end{itemize}

\section{Nieuw project}
\subsection{Webpack}
Een nieuw project opzetten met Webpack kan op verschillende manieren: zelf een project opzetten en de configuratie volledig manueel schrijven of gebruik maken van een framework waarin het al geconfigureerd voor ons is. Aangezien niet velen het eerste pad bewandelen, zullen we gebruik maken van de meest populaire manier om een React project op te zetten. Create-react-app of CRA is een minimale framework gemaakt door de makers van React zelf om gemakkelijk een React omgeving op te zetten. Het is één van de vele frameworks die Webpack als module bundler gebruikt. Om te beginnen, voeren we volgende commando’s uit. 

\lstinputlisting[language=bash]{codeSnippets/craCreate.txt}

Bovenstaande code maakt een project aan met create-react-app. Daarna ziet ons project er als volgt uit. Merk op dat er geen config bestand voor Webpack is.

\begin{figure}[h]
\includegraphics{fileStructureInitCRA}
    \centering
    \caption{Bestandsstructuur nieuw CRA project}
\end{figure}

Alles wat in de public map staat, zijn statische assets. Die zullen dus via een url bereikbaar zijn. Als we kijken in het index.html bestand, merken we op dat er geen enkele script-tag aanwezig is. Hierover later meer.
In de src map staat alles wat door Webpack zal gebundeld worden. Standaard worden er CSS bestanden aangemaakt voor styling en een logo in svg formaat. Dit wordt gedaan om aan te tonen dat we deze assets gewoon in de Javascript code kunnen importeren, de bundler kan hiermee overweg. 

\lstinputlisting[language=Javascript]{codeSnippets/importCSS.js}

In het package.json bestand staat er allemaal info over dit project. In het dependencies gedeelte staan alle externe packages die dit project gebruikt. Nieuwe packages worden gedownload aan de hand van Node Package Manager of NPM. Bij de dependencies staat een package genaamd “react-scripts”. 
React-scripts \autocite{facebook-2018} is een package gemaakt door de makers van React en is de motor achter CRA. 
Wat voor dit onderzoek relevant is, is dat het de Webpack.config bevat. Dit bestand bestaat uit maar liefst 700+ lijnen code \autocite{facebook-2021}. Nu is de reden dat we CRA gebruiken en niet van nul beginnen, duidelijk. 

De volgende stap is om het project lokaal op te starten. React-scripts gebruikt de ingebouwde development server van Webpack. Na het uitvoeren van volgend commando, wordt die server opgestart en opent de webapplicatie in een browser.

\lstinputlisting[language=bash]{codeSnippets/craStart.txt}

\begin{table}[h]
    \centering
    \begin{tabular}{lr}
    \textbf{Grootte project (MB)} & 0,037 \\
    \textbf{Grootte node\_modules (MB)} & 214,1 \\
    \textbf{Grootte uitvoer (MB)} & 0,514 \\
    \textbf{Snelheid creatie uitvoer (s)} & 3,67 \\
    \textbf{} & 5 \\
    \textbf{} & 3 \\
    \textbf{} & 3 \\
    \textbf{Snelheid opstarten development server (s)} & 2,67 \\
    \textbf{} & 4 \\
    \textbf{} & 2 \\
    \textbf{} & 2
    \end{tabular}
    \caption{Overzicht nieuw project met Webpack}
    \end{table}

\subsection{Parcel}
Bij Webpack gebruikten we een framework om het vele configuratie werk te omzeilen. Bij Parcel is dit niet nodig. Zoals in de literatuurstudie vermeld werkt Parcel op een gelijkaardige manier als Webpack, maar dan met zo min mogelijk configuratie. Om een nieuw project op te zetten, gaan we dus geen framework gebruiken.

Maak een nieuwe map aan waar het project zal leven. Aangezien we van nul beginnen, moeten de bestanden uit figuur 3.2 zelf aangemaakt worden.

\begin{figure}[h]
    \includegraphics{parcelNew}
        \centering
        \caption{Aan te maken bestanden voor nieuw Parcel project}
    \end{figure}

    Hierna moeten er nog enkele packages geïnstalleerd worden, namelijk: React, React-DOM en natuurlijk Parcel. In de package.json moet er ook nog meegegeven worden waar de index.html zich bevindt. Merk op dat er geen apart configuratiebestand voor Parcel is. Nu kan het project opgestart worden met hetzelfde commando als bij Webpack.

    Merk op dat er geen public folder aanwezig is zoals in CRA. Er is momenteel geen folder waar statische bestanden kunnen leven. Als dit een vereiste is, heeft Parcel een plugin nodig. Gelukkig zijn die gemakkelijk te installeren.

\subsection{Snowpack}
Voor Snowpack gaan we net zoals bij Parcel te werk zonder framework. In tegenstelling tot Parcel heeft het Snowpack team al een speciaal react-template gemaakt met een kant en klaar commando om het te initialiseren. Zelf de bestanden aanmaken en dependencies toevoegen is dus niet nodig.

\lstinputlisting[language=bash]{codeSnippets/snowpackNew.txt}
\begin{figure}[h]
    \includegraphics{snowpackNew}
        \centering
        \caption[Aangemaakte bestanden door Snowpack commando]{Aangemaakte bestanden door commado}
    \end{figure}

Deze structuur is identiek aan die van CRA. In tegenstelling tot Parcel is een public folder al geconfigureerd waar statische bestanden, zoals afbeeldingen, beschikbaar staan. Een configuratiebestand voor Snowpack is ook aangemaakt. Hierin staan al 25 lijnen configuratie voor ons geschreven. Meer dan Parcel maar aanzienlijk minder dan CRA. Merk op dat sommige bestanden nu de extensie .jsx in plaats van .js hebben. Dit komt doordat Snowpack geen JSX tolereert in .js bestanden. JSX is wat een React component retourneert i.e elk React component moet een .jsx extensie hebben. Geen probleem bij een nieuw project maar bij een oud kan het nodig zijn om vele bestanden van extensie te veranderen, zie later.

Zoals in de literatuurstudie vermeld, is Snowpack geen module bundler aangezien het de verschillende bestanden in een project niet bundelt. In productie kan dat optioneel nog gedaan worden door Webpack of Rollup maar dat is niet standaard. In development heeft dit het grote voordeel dat de bundel niet telkens opnieuw opgebouwd moet worden als een bestand veranderd. 

\subsection{Vite}
Vite is nog een voorbeeld van een unbundled build tool, net zoals Snowpack. Een nieuw project opzetten is heel gemakkelijk aan de hand van een simpel commando dat ze voorzien hebben, net zoals Snowpack. 

\lstinputlisting[language=bash]{codeSnippets/viteNew.txt}

\begin{figure}[h]
    \includegraphics{viteNew}
        \centering
        \caption[Aangemaakte bestanden door Vite commando]{Aangemaakte bestanden door commado}
    \end{figure}

Er is een klein config bestand aanwezig van 7 lijnen code waar de react plugin is geïnitialiseerd. Voor de rest ziet het project in grote lijnen er uit als dat van Snowpack. Er is geen public map aanwezig maar dat kan aangemaakt worden en wordt automatisch geconfigureerd.

\section{Bestaand project}
In het vorige deel, werden nieuwe projecten opgezet voor de respectievelijke module bundlers. In dit deel worden bestaande projecten, opgezet met Webpack, omgevormd zodat ze gebruik maken van de andere opties. Per project wordt eerst wat toelichting gegeven en vervolgens per module bundler wat er nodig is om ze werkende te krijgen. 

\subsection{Mortage}
Het eerste project dat we gaan proberen omvormen is de Mortage Overpayment Calculator \autocite{houghton-2019}.
Het is een zeer simpel en klein project van 17 KB groot dat is opgezet met CRA. 
Daarnaast maakt het nog gebruik van andere open-source packages die verzameld worden in de nodemodules map. 
Die map is 214 MB groot. 

Mortage is een ideaal project om mee te starten aangezien het geen speciale technologieën gebruikt. CSS voor stijl, wat de browser begrijpt zonder enige omvorming en voor de rest Javascript en HTML. Normaal zouden we dus niet in de problemen mogen komen. Een potentiële moeilijkheid wat dit project ook vermijdt is statische bestanden. 

\begin{figure}[h]
    \includegraphics{Mortage_languages}
        \centering
        \caption{Overzicht gebruikte technologieën van het Mortage project}
    \end{figure}

\subsubsection{Parcel}
Zoals in de literatuurstudie vermeld, probeert Parcel hetzelfde als Webpack te bereiken maar met veel minder tot geen configuratie. Die bewering zal nu nagegaan worden. Mortage is opgezet met CRA en gebruikt dus het react-scripts package die onder andere alle Webpack configuratie op zich neemt. Om Webpack in te ruilen voor Parcel moeten we dus eerst react-scripts bij het grofvuil zetten. Het package.json bestand bevat alle info over een project: de naam, versie, welke andere packages het gebruikt, hoe het wordt opgestart en nog veel meer. Ook Mortage heeft zo’n bestand en dat ziet er als volgt uit:

\lstinputlisting[language=Javascript]{codeSnippets/packageMortage.json}

In het gedeelte “scripts” staan de verschillende commando’s. Het start en build commando starten het project in development en production modus respectievelijk op. Beide voeren op hun beurt een commando van react-scripts uit. Dit gaat dus vervangen moeten worden. In de documentatie van Parcel valt te lezen dat we die commando’s moeten vervangen met de volgende:

\lstinputlisting[language=Javascript]{codeSnippets/scriptsParcelMortage.json}

Vervolgens moet het index.html bestand aangepast worden. Voor de wijzigingen zag het er als volgt uit:

\lstinputlisting[language=HTML]{codeSnippets/mortageIndex.html}

Merk op dat er nergens een verwijzing is naar een Javascript bestand. React-scripts voegt dat automatisch toe bij het bouwen van het project. Parcel doet dat niet dus er moet nog een expliciete verwijzing komen. 

\lstinputlisting[language=HTML]{codeSnippets/mortageIndexScript.html}

Nu moet Parcel uiteraard nog gedownload worden. Nadien werkt het project. 

\subsubsection{Snowpack}
Snowpack gebruikt de principes van unbundled development uitgelegd in de literatuurstudie. Wat dat betekent in de werkelijkheid volgt later. Eerst kijken we hoe Mortage kan omgevormd worden van Webpack naar Snowpack. 

Bij Parcel lag de focus op zo weinig mogelijk configuratie, niet bij Snowpack. Snowpack is veel moeilijker te configureren dan Parcel en Vite en voelt in dat opzicht aan als Webpack. We ondernemen dezelfde stappen als hierboven om react-scripts te verwijderen en die te vervangen door de nieuwe commando’s. Daarna voegen we Snowpack toe samen met twee andere plugins voor React. We doen net dezelfde wijzigingen aan index.html, voegen een configuratie bestand toe en proberen de app te runnen. Een foutmelding verschijnt. Om die te kunnen begrijpen, is er eerst wat meer uitleg nodig.

Een React component retourneert JSX. JSX is, zonder te veel in details te gaan, een extensie bovenop Javascript die een manier biedt om de weergave van componenten te structureren en lijkt heel hard op HTML. Het belangrijke om hiervan mee te nemen is dat het een Javascript extensie is. In CRA en sommige andere frameworks is het mogelijk om JSX in een bestand te schrijven met de standaard Javascript extensie .js . Snowpack ondersteunt dit niet. Als een bestand JSX bevat, moet die de extensie .jsx krijgen. Dus moeten we alle bestanden die JSX gebruiken van bestandsextensie veranderen. Hierna werkt de app naar behoren. 

\subsubsection{Vite}

Vite combineert het beste van de twee voorgaande build tools. Het gebruikt de principes van unbundled development en dat met weinig configuratie. Om Mortage op te zetten met Vite overlopen we stappen die eerder al aan bod kwamen: react-scripts verwijderen, index.html en package.json naar de nieuwe commando’s aanpassen en Vite zelf installeren. Net zoals Snowpack ondersteunt Vite geen JSX in een .js bestand en is er een plugin voor React. In tegenstelling tot Snowpack is de plugin niet nodig om het project werkend te krijgen. Na het aanpassen van de bestandsextensies en eventueel downloaden van de plugin, kan dit project zonder problemen opgestart worden. 

\subsection{Todoist}

Todoist \autocite{hadwen-2021} is een simpele to-do webapplicatie. Een gebruiker kan taken toevoegen die nog moeten gedaan worden en aanduiden wanneer die voltooid zijn. Het is een groter project dan Mortage met een grootte van 106KB en 377MB nodemodules maar kan nog steeds beschouwd worden als een kleine tot medium grootte app. Het heeft een database nodig om de taken te kunnen opslaan en zoals hieronder te zien is, gebruikt het een andere taal voor stijl: SCSS. SCSS is een superset i.e. een uitbreiding van gewone CSS. Aangezien een webbrowser die uitbreiding niet ondersteunt, moet het SCSS bestand omgevormd worden naar gewone CSS wanneer de applicatie start. Dit kan eventueel tot extra configuratie leiden bij het instellen van een nieuwe build tool. 

\begin{figure}[h]
    \includegraphics{Todoist_languages}
        \centering
        \caption{Overzicht gebruikte technologieën van het Todoist project}
    \end{figure}

\subsubsection{Parcel}

Om dit project om te vormen naar Parcel worden eerst dezelfde stappen overlopen als in het vorige Parcel project. Er zijn twee mogelijke extra struikelblokken bij Todoist: het gebruik van SCSS en statische bestanden. Het eerste overkomt Parcel met glans: we moeten niets zelf configureren. De eerste keer dat Todoist wordt gestart, detecteert Parcel dat er SCCS aanwezig is en download de bijhorende plugin. Het tweede probleem vereist ook een plugin en daarenboven nog wat configuratie. In de package.json duiden we met volgende lijnen aan waar het mapje met statische bestanden zich bevindt. Nadien moeten de verwijzingen naar de afbeeldingen nog veranderd worden naar de nieuwe locatie. Hierna is ook dit project bijna klaar voor gebruik. Enkel nog het configuratie bestand voor de database moet nog aangepast worden.

\subsubsection{Snowpack}

Ook hier zijn de stappen gelijkaardig aan het vorig project. Na die te doorlopen volgen nog twee potentiële moeilijkheden. In het vorig deel over Snowpack werd al vermeld dat statistische bestanden ondersteunt worden zonder extra plugin. Het tweede probleem, SCSS, kan eveneens opgelost worden door een plugin. In tegenstelling tot Parcel, moet die wel manueel toegevoegd worden. 

\subsubsection{Vite}

Voor Vite kunnen we heel kort blijven. Na het uitvoeren van dezelfde stappen als vorig project, werkt alles. Zowel statische bestanden als SCCS zijn ondersteunt zonder extra plugins of configuratie.

\subsection{Bar}
Als laatste nemen we het grootste en meest complexe project van deze studie onder de loep. De BarApp \autocite{vansteenkiste-2021} is ontworpen om barlijsten bij te houden zonder zelf rekening te houden met de effectieve betaling. In vergelijking met de voorgaande projecten is het veel groter: 27MB voor het project zelf en dan nog eens 442MB voor de nodemodules. SCCS valt weg als moeilijkheid en wordt vervangen door twee nieuwe: TypeScript en Tailwind CSS.

\begin{figure}[h]
    \includegraphics{Barapp_languages}
        \centering
        \caption{Overzicht gebruikte technologieën van het BarApp project}
    \end{figure}

    TypeScript is net als SCCS een superset of uitbreiding. Deze keer niet van CSS maar van Javascript. De browser begrijpt geen TypeScript dus moet het eerst vertaald worden naar normale Javascript. De build tools zijn hier verantwoordelijk voor.

    De tweede moeilijkheid, Tailwind CSS, staat niet in het gebruikte-talen-overzicht van hierboven. Toch is het iets waar we rekening mee moeten houden. Het is een collectie van allemaal CSS klassen die de gebruiker direct in een JSX element kan gebruiken. Tailwind zelf bevat dus hele grote CSS bestanden om die klassen allemaal te definiëren. Aangezien we enkel de klassen willen behouden die effectief gebruikt worden in het project, worden de ongebruikte gewist bij het bouwen van de applicatie voor productie. Hoewel dit alles geen taak is van de build tool zelf, kunnen er moeilijkheden optreden bij de configuratie.
    
\subsubsection{Parcel}

Nog maar eens overkomt Parcel de mogelijke struikelblokken met zo goed als geen configuratie. Het ondersteunt Typescript zonder iets extra te moeten doen. Naast de standaard configuratie van Tailwind zelf, is er niets anders nodig om het werkend te krijgen. Na het overlopen van de stappen van de vorige Parcel projecten, met in het bijzonder de stap voor statische bestanden, is ook deze applicatie bijna klaar voor gebruik. Voor het laatste detail nemen we nog een kijkje naar de index.html. Aangezien CRA op een andere manier verwijst naar de afbeeldingen die hier staan, moeten de href attributen aangepast worden.

\subsubsection{Snowpack}
Om al een tipje van de sluier op te lichten: dit project was veel lastiger om op te zetten met Snowpack dan met de andere twee opties. Eerst het goede nieuws: Typescript is ook ondersteund zonder extra configuratie. Daar stopt de vreugde al. De overige problemen worden door hun complexiteit in paragrafen onderverdeeld. 

Voor Tailwind CSS valt het nog wel mee. Naast dezelfde configuratie en installatie stappen die moeten doorlopen worden als bij Parcel (en Vite), moet er nog een plugin geïnstalleerd worden voor Snowpack zelf. Hierna moet die zijn configuratie bestand aangepast worden zodat de plugin kan werken. Dat doen we door op twee verschillende plaatsen een lijn code toe te voegen. 

Zoals in de literatuurstudie uitgelegd, ondersteunen moderne browsers enkel ESM. Aangezien Snowpack een unbundled build tool is, stuurt het de bestanden van het project naar de browser zonder ze te bundelen. 

\begin{figure}[h]
    \includegraphics{bundledVSUnbundledFiles}
        \centering
        \caption{Bestandsstructuur unbundled vs bundled}
    \end{figure}

    Merk op dat bij de unbundled versie van dit project, de bestandsstructuur dezelfde is als hoe we het project maken. Een bundler zoals Webpack, steekt alle bestanden die naar elkaar verwijzen samen in één bestand dat dan naar de browser gestuurd wordt. Unbundled build tools zoals Snowpack laten de bestandsstructuur met rust omdat ze vertrouwen op het feit dat moderne browsers ESM ondersteunen. Dus zolang er in het project enkel ESM wordt gebruikt, is er geen probleem. Jammer genoeg staat er in de App.jsx een require functie, een CommonJS module dus. Omdat Snowpack niets doet om dit te vertalen zal de CommonJS module manueel omgevormd moeten worden naar ESM. Gelukkig wordt die vertaling wel uitgevoerd voor de nodemodules, want die bevatten ook vaak CommonJS.

    Voor het volgend probleem, wordt het index.jsx bestand onder de loep genomen.

    \lstinputlisting[language=Javascript]{codeSnippets/indexBar.js}

    De aangeduide code vormt een probleem. De lijn process.env.NODE env === “production” wordt gebruikt om te kijken of de app in productie draait of lokaal, in development. Naast dit voorbeeld wordt deze methode vaak toegepast in dit project. Het probleem is dat Snowpack het niet ondersteunt. Omdat dit voorbeeld niet cruciaal is, laten we het weg om Snowpack opgezet te krijgen. Jammer genoeg gebruikt deze app Workbox. Workbox is een Javascript library die een PWA opzetten, gemakkelijker maakt. Een PWA is een webapplicatie die installeerbaar is op een Smartphone of computer. Het probleem is dat Workbox ook gebruikt maakt van bovenstaande methode. Als we deze library nog steeds zouden willen gebruiken, zou dit alles ook moeten herschreven worden. Voorlopig gaan we verder zonder aangezien het veel tijd in beslag zou nemen. Op naar het laatste probleem.

    React kan al even gebruikt worden zonder het te moeten importeren in een bestand. Dat wordt automatisch voor de developer gedaan. Jammer maar helaas: Snowpack ondersteunt dit niet. Aan elk Javascript bestand waar iets van de React methodes gebruikt worden, moeten een import statement toevoegen als volgt:

    \lstinputlisting[language=Javascript]{codeSnippets/importReact.js}

    Na dit alles te overlopen, werkt het project. Eindelijk.

\subsubsection{Vite}
Na de hele waslijst problemen die bij Snowpack opdoken, doet het deugd om weer met Vite aan de slag te mogen gaan. Van de struikelblokken die Snowpack tegenkwam, blijft voor Vite enkel nog het probleem van de CommonJS module over. Na die te herschrijven en dezelfde stappen te doorlopen als bij de vorige Vite secties, werkt dit project volledig, ook Workbox. Typescript wordt ook ondersteunt zonder extra gedoe en voor Tailwind is de gebruikelijke installatie en configuratie nodig, geen extra plugin. Net zoals bij Parcel moeten de verwijzingen naar de afbeeldingen in de index.html bijgeschaafd worden. 

De conclusie over de bevindingen die in dit hoofdstuk waargenomen zijn, volgt in het volgend hoofdstuk. 


% Voeg hier je eigen hoofdstukken toe die de ``corpus'' van je bachelorproef
% vormen. De structuur en titels hangen af van je eigen onderzoek. Je kan bv.
% elke fase in je onderzoek in een apart hoofdstuk bespreken.

%\input{...}
%\input{...}
%...

\chapter{Conclusie}
\label{ch:conclusie}

In de twee vorige hoofdstukken werden de gekozen build tools op de proef gesteld aan de hand van drie open-source projecten en een van nul te beginnen. De verschillende stappen om ze werkende te krijgen werden overlopen, bij de een waren dat er al meer dan de ander. Bij elk van die twee hoofdstukken werd op het einde ook een aparte conclusie getrokken. Dit hoofdstuk dient als algemene conclusie. 

De hoofdvraag van deze proef, of Webpack nog steeds mag gezien worden als de koning der Javascript build tools, kan beantwoord worden met een duidelijke ‘nee’. Zelf binnen zijn eigen categorie van module bundlers, blijkt Parcel, in de meeste gevallen, een betere optie te zijn. Vite bewees dat de ongebundelde build tools de toekomst zijn. Module bundlers zijn hierdoor echter nog niet voorbijgestreefd. Zelfs wanneer browsers en Node.js ESModules volledig ondersteunen en alle open-source packages er volledig mee geschreven zijn, zullen de voordelen van module bundlers hun plaats in de Javascript wereld garanderen. Voor de nabije toekomst zullen module bundlers en ongebundelde build tools hand in hand moeten samenwerken om het beste van beide werelden samen te brengen. Vite lijkt dit het best te begrijpen. Het is dus wel duidelijk dat Vite de beste optie is voor zowel een nieuwe webapplicatie te ontwikkelen, als een bestaand een nieuwe, snellere en modernere motor te geven. 



%%=============================================================================
%% Bijlagen
%%=============================================================================

\appendix
\renewcommand{\chaptername}{Appendix}

%%---------- Onderzoeksvoorstel -----------------------------------------------

\chapter{Onderzoeksvoorstel}

Het onderwerp van deze bachelorproef is gebaseerd op een onderzoeksvoorstel dat vooraf werd beoordeeld door de promotor. Dat voorstel is opgenomen in deze bijlage.

% Verwijzing naar het bestand met de inhoud van het onderzoeksvoorstel
%---------- Inleiding ---------------------------------------------------------

\section{Introductie} % The \section*{} command stops section numbering
\label{sec:introductie}

A module bundler is one of the most essential technologies in modern web development. Nobody makes webapps with plain HTML, CSS and JS anymore. So the choice of which bundler to use is very important.
While Webpack is the most popular there are many others that claim to do the job better than it. Why is it still the king then I wonder. How does Webpack compare to its competitors? Are their claims true? If so, why is Webpack still the most popular? With this paper. 

Many developers give much thought about which framework to use, which back-end and so forth. The module bundler however often doesn't get that much thoutgh. This is because it's less attractive and in many frameworks comes pre-installed. But as your webapp can't run without it, I believe . In this paper, I want to demystify the module bundler and compare the most popular options. Not just in terms of speed and module size, but also its plugins, developer experience, ... .

Hier introduceer je werk. Je hoeft hier nog niet te technisch te gaan.

Je beschrijft zeker:

\begin{itemize}
  \item de probleemstelling en context
  \item de motivatie en relevantie voor het onderzoek
  \item de doelstelling en onderzoeksvraag/-vragen
\end{itemize}

%---------- Stand van zaken ---------------------------------------------------

\section{State-of-the-art}
\label{sec:state-of-the-art}

Much has already been written about module bundlers online. While they provide very useful information, I have yet to find an in-depth analysis of the differences between the major players and what that means for the developer and the output product.

Hier beschrijf je de \emph{state-of-the-art} rondom je gekozen onderzoeksdomein. Dit kan bijvoorbeeld een literatuurstudie zijn. Je mag de titel van deze sectie ook aanpassen (literatuurstudie, stand van zaken, enz.). Zijn er al gelijkaardige onderzoeken gevoerd? Wat concluderen ze? Wat is het verschil met jouw onderzoek? Wat is de relevantie met jouw onderzoek?

Verwijs bij elke introductie van een term of bewering over het domein naar de vakliteratuur, bijvoorbeeld~\autocite{Doll1954}! Denk zeker goed na welke werken je refereert en waarom.

% Voor literatuurverwijzingen zijn er twee belangrijke commando's:
% \autocite{KEY} => (Auteur, jaartal) Gebruik dit als de naam van de auteur
%   geen onderdeel is van de zin.
% \textcite{KEY} => Auteur (jaartal)  Gebruik dit als de auteursnaam wel een
%   functie heeft in de zin (bv. ``Uit onderzoek door Doll & Hill (1954) bleek
%   ...'')

Je mag gerust gebruik maken van subsecties in dit onderdeel.

%---------- Methodologie ------------------------------------------------------
\section{Methodologie}
\label{sec:methodologie}

First I will compare the most popular bundlers theorethicly. How they work and how they may differ.
To compare them in practice will require code-bases of many sizes. Why many sizes, you ask? Because it could be interesting to see how module bundlers deal with smaller projects and larger ones. To see if the advantages of one deminish when the project size in- or decreases. 

I will test each codebase with many bundlers and then observe the differences in numerous categories. One important of which is the developer experience: how easy is it to install? How simple or compley is the config file and the plugins. There are a lot of things to consider  
Hier beschrijf je hoe je van plan bent het onderzoek te voeren. Welke onderzoekstechniek ga je toepassen om elk van je onderzoeksvragen te beantwoorden? Gebruik je hiervoor experimenten, vragenlijsten, simulaties? Je beschrijft ook al welke tools je denkt hiervoor te gebruiken of te ontwikkelen.

%---------- Verwachte resultaten ----------------------------------------------
\section{Verwachte resultaten}
\label{sec:verwachte_resultaten}

Many module bundlers claim speedier builds and smaller bundle sizes. If that's the case for a project of any size remains to be seen. But I do expect to see a difference there. The question will be if those improvements warrant the competitor to be used over Webpack. Developer experience and performance in development mode are defining factors in my opinion as well. Obviously the former will be quite subjective so that's an important factor to note.

%---------- Verwachte conclusies ----------------------------------------------
\section{Verwachte conclusies}
\label{sec:verwachte_conclusies}

I honestly have no idea what I will conclude. That's the most important reason why I want to conduct this research. Webpack has served us well but as the tech world shifts more and more towards applications written with web technologies, the time of newer approaches may have come.



%%---------- Andere bijlagen --------------------------------------------------
% TODO: Voeg hier eventuele andere bijlagen toe
%\input{...}

%%---------- Referentielijst --------------------------------------------------

\printbibliography[heading=bibintoc]

\end{document}
