%---------- Inleiding ---------------------------------------------------------

\section{Introduction} % The \section*{} command stops section numbering
\label{sec:introductie}

A module bundler is one of the most essential technologies in modern web development. Nobody makes web applications with plain HTML, CSS and JS anymore. All the fancy new tools, frameworks and libraries for a web developer to use, like Angular or React, won't work in the browser without being bundled. So the choice of which bundler to use is very important.
While Webpack is the most popular\textsuperscript{1,2,3}, there are many others that claim to do the job better. How does Webpack compare to its competitors? Are their claims valid? If so, why is Webpack still the most popular? Thus to summarize: does it make sense for development teams at companies, stand-alone developers or just hobby coders to bundle their code with the same tool as 5 years ago? 

Many developers give much thought about which framework to use, which back-end and so forth. The module bundler however often doesn't get that much thought. This is because it's less attractive and in many frameworks comes pre-installed. But as your web app can't run without it, it's one of the most important choices you make. The goal of this paper is to demystify the module bundler and compare the most popular options. Not just in terms of speed and module size, but also its plugins, developer experience, ... .


%---------- Stand van zaken ---------------------------------------------------

\section{State-of-the-art}
\label{sec:state-of-the-art}

Much can already be found about module bundlers online. On the website or GitHub repository of the bundler itself, articles written by third-parties and video's on YouTube. While they contain very useful information, they lack an in-depth analysis of the differences between the major players and what that means for the developer and the output product. This is an important goal of the paper.


%---------- Methodologie ------------------------------------------------------
\section{Methodology}
\label{sec:methodologie}

Firstly, the most popular bundlers will be campered theoretically. How they work and how they may differ. 
To compare them in practice will require code-bases of many sizes. Luckily there are many open-source codebases on GitHub to chose from. Why many sizes, you ask? Because it could be interesting to see how module bundlers deal with smaller projects compared to larger ones. To see if the advantages of one fade when the project size in- or decreases. It could also be interesting to compare them based on the framework, like Angular or Create-React-App. Those frameworks ship with pre-configured bundlers so it remains to be seen if it's even possible to replace those with others. It's something that has to tested.

Secondly and maybe more important: what about already existing, operational projects? For example: a company that already has a site or some kind of application on the web bundled with WebPack. Is it possible to just switch to another module bundler? What hurdles have to be overcome to do so and what are the benefits? 

Testing many code-bases and observing the differences in numerous categories is necessary. One important category is the developer experience: how easy is it to install? How simple or complex are the config file and plugins. There are a lot of things to consider. 

%---------- Verwachte resultaten ----------------------------------------------
\section{Anticipated results}
\label{sec:verwachte_resultaten}

Many module bundlers claim speedier builds and smaller bundle sizes. So that's an easy prediction to make. Wether that remains true with projects of any size remains to be seen. Will those improvements be great enough to warrant the competitor to be used over Webpack? Developer experience and performance in development mode are defining factors as well. Obviously the former will be quite subjective so that's an important factor to note.

%---------- Verwachte conclusies ----------------------------------------------
\section{Anticipated conclusions}
\label{sec:verwachte_conclusies}

It's hard to predict what the conclusion will be. That's the most important reason why I want to conduct this research. Webpack has served us well but as the tech world shifts more and more towards applications written with web technologies, the time of newer approaches may have come.

%---------- Verwachte conclusies ----------------------------------------------
\section{References}
\label{sec:references}

\begin{enumerate}
    \item Search results module bundlers. (n.d.). Google Trends. Retrieved October 2021, from \href{https://trends.google.com/trends/explore?date=all&geo=BE&q=\%2Fg\%2F11clg_kyfc,\%2Fm\%2F02qkmx0,\%2Fm\%2F012ccfhw}{https://trends.google.com}
    \item Nalakath, N. (2021, February 2). Module Bundlers and their role in web development. | Better Programming. Medium. Retrieved October 2021, from https://betterprogramming.pub/javascript-module-bundlers-2a1e9307d057
    \item The State of JavaScript 2018: Other Tools. (2018). StateOfJs. Retrieved October 2021, from https://2018.stateofjs.com/other-tools/
\end{enumerate}

