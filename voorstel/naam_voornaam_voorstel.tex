%==============================================================================
% Sjabloon onderzoeksvoorstel bachelorproef
%==============================================================================
% Gebaseerd op LaTeX-sjabloon ‘Stylish Article’ (zie voorstel.cls)
% Auteur: Jens Buysse, Bert Van Vreckem
%
% Compileren in TeXstudio:
%
% - Zorg dat Biber de bibliografie compileert (en niet Biblatex)
%   Options > Configure > Build > Default Bibliography Tool: "txs:///biber"
% - F5 om te compileren en het resultaat te bekijken.
% - Als de bibliografie niet zichtbaar is, probeer dan F5 - F8 - F5
%   Met F8 compileer je de bibliografie apart.
%
% Als je JabRef gebruikt voor het bijhouden van de bibliografie, zorg dan
% dat je in ``biblatex''-modus opslaat: File > Switch to BibLaTeX mode.

\documentclass{voorstel}

\usepackage{lipsum}

%------------------------------------------------------------------------------
% Metadata over het voorstel
%------------------------------------------------------------------------------

%---------- Titel & auteur ----------------------------------------------------

% TODO: geef werktitel van je eigen voorstel op
\PaperTitle{Can Webpack still be considered the golden standard of module bundlers?}
\PaperType{Research proposal bachelor’s thesis 2021-2022} % Type document

% TODO: vul je eigen naam in als auteur, geef ook je emailadres mee!
\Authors{Maxim Vansteenkiste\textsuperscript{1}} % Authors
\CoPromotor{Cedric Vanhaeverbeke\textsuperscript{2} (CodeFever)}
\affiliation{\textbf{Contact:}
  \textsuperscript{1} \href{mailto:maxim.vansteenkiste@student.hogent.be}{maxim.vansteenkiste@student.hogent.be};
  \textsuperscript{2} \href{mailto:cedric@codefever.be}{cedric@codefever.be};
}

%---------- Abstract ----------------------------------------------------------

\Abstract{The simple times of creating a website with plain HTML, CSS and JS are long gone. Nowadays we want to use a UI library like React or a CSS pre-compiler like Sass. All great technologies that make the life of a web developer easier and allow them to create even more awesome apps. But all these dependencies and different filetypes need to be bundled for a browser to understand them. That's exactly what a module bundler does. There are many module bundlers but by far the most popular is Webpack.
Why is this the case? How does it compare to the competition? Should it still be considered as the golden standard? All of these questions will be answered in this paper where I take a deep-dive into the world of module bundlers.}

%---------- Onderzoeksdomein en sleutelwoorden --------------------------------
% TODO: Sleutelwoorden:
%
% Het eerste sleutelwoord beschrijft het onderzoeksdomein. Je kan kiezen uit
% deze lijst:
%
% - Mobiele applicatieontwikkeling
% - Webapplicatieontwikkeling
% - Applicatieontwikkeling (andere)
% - Systeembeheer
% - Netwerkbeheer
% - Mainframe
% - E-business
% - Databanken en big data
% - Machineleertechnieken en kunstmatige intelligentie
% - Andere (specifieer)
%
% De andere sleutelwoorden zijn vrij te kiezen

\Keywords{Module bundler --- Web development --- Javascript} % Keywords
\newcommand{\keywordname}{Keywords} % Defines the keywords heading name

%---------- Titel, inhoud -----------------------------------------------------

\begin{document}

\flushbottom % Makes all text pages the same height
\maketitle % Print the title and abstract box
\tableofcontents % Print the contents section
\thispagestyle{empty} % Removes page numbering from the first page

%------------------------------------------------------------------------------
% Hoofdtekst
%------------------------------------------------------------------------------

% De hoofdtekst van het voorstel zit in een apart bestand, zodat het makkelijk
% kan opgenomen worden in de bijlagen van de bachelorproef zelf.
%---------- Inleiding ---------------------------------------------------------

\section{Introductie} % The \section*{} command stops section numbering
\label{sec:introductie}

A module bundler is one of the most essential technologies in modern web development. Nobody makes webapps with plain HTML, CSS and JS anymore. So the choice of which bundler to use is very important.
While Webpack is the most popular there are many others that claim to do the job better than it. Why is it still the king then I wonder. How does Webpack compare to its competitors? Are their claims true? If so, why is Webpack still the most popular? With this paper. 

Many developers give much thought about which framework to use, which back-end and so forth. The module bundler however often doesn't get that much thoutgh. This is because it's less attractive and in many frameworks comes pre-installed. But as your webapp can't run without it, I believe . In this paper, I want to demystify the module bundler and compare the most popular options. Not just in terms of speed and module size, but also its plugins, developer experience, ... .

Hier introduceer je werk. Je hoeft hier nog niet te technisch te gaan.

Je beschrijft zeker:

\begin{itemize}
  \item de probleemstelling en context
  \item de motivatie en relevantie voor het onderzoek
  \item de doelstelling en onderzoeksvraag/-vragen
\end{itemize}

%---------- Stand van zaken ---------------------------------------------------

\section{State-of-the-art}
\label{sec:state-of-the-art}

Much has already been written about module bundlers online. However many don't go in depth with the workings of it or stay very superficial. While they provide very useful information, I

Hier beschrijf je de \emph{state-of-the-art} rondom je gekozen onderzoeksdomein. Dit kan bijvoorbeeld een literatuurstudie zijn. Je mag de titel van deze sectie ook aanpassen (literatuurstudie, stand van zaken, enz.). Zijn er al gelijkaardige onderzoeken gevoerd? Wat concluderen ze? Wat is het verschil met jouw onderzoek? Wat is de relevantie met jouw onderzoek?

Verwijs bij elke introductie van een term of bewering over het domein naar de vakliteratuur, bijvoorbeeld~\autocite{Doll1954}! Denk zeker goed na welke werken je refereert en waarom.

% Voor literatuurverwijzingen zijn er twee belangrijke commando's:
% \autocite{KEY} => (Auteur, jaartal) Gebruik dit als de naam van de auteur
%   geen onderdeel is van de zin.
% \textcite{KEY} => Auteur (jaartal)  Gebruik dit als de auteursnaam wel een
%   functie heeft in de zin (bv. ``Uit onderzoek door Doll & Hill (1954) bleek
%   ...'')

Je mag gerust gebruik maken van subsecties in dit onderdeel.

%---------- Methodologie ------------------------------------------------------
\section{Methodologie}
\label{sec:methodologie}

Hier beschrijf je hoe je van plan bent het onderzoek te voeren. Welke onderzoekstechniek ga je toepassen om elk van je onderzoeksvragen te beantwoorden? Gebruik je hiervoor experimenten, vragenlijsten, simulaties? Je beschrijft ook al welke tools je denkt hiervoor te gebruiken of te ontwikkelen.

%---------- Verwachte resultaten ----------------------------------------------
\section{Verwachte resultaten}
\label{sec:verwachte_resultaten}

Hier beschrijf je welke resultaten je verwacht. Als je metingen en simulaties uitvoert, kan je hier al mock-ups maken van de grafieken samen met de verwachte conclusies. Benoem zeker al je assen en de stukken van de grafiek die je gaat gebruiken. Dit zorgt ervoor dat je concreet weet hoe je je data gaat moeten structureren.

%---------- Verwachte conclusies ----------------------------------------------
\section{Verwachte conclusies}
\label{sec:verwachte_conclusies}

Hier beschrijf je wat je verwacht uit je onderzoek, met de motivatie waarom. Het is \textbf{niet} erg indien uit je onderzoek andere resultaten en conclusies vloeien dan dat je hier beschrijft: het is dan juist interessant om te onderzoeken waarom jouw hypothesen niet overeenkomen met de resultaten.



%------------------------------------------------------------------------------
% Referentielijst
%------------------------------------------------------------------------------
% TODO: de gerefereerde werken moeten in BibTeX-bestand ``voorstel.bib''
% voorkomen. Gebruik JabRef om je bibliografie bij te houden en vergeet niet
% om compatibiliteit met Biber/BibLaTeX aan te zetten (File > Switch to
% BibLaTeX mode)

\phantomsection
\printbibliography[heading=bibintoc]

\end{document}
