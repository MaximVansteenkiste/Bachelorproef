%---------- Inleiding ---------------------------------------------------------

\section{Introductie} % The \section*{} command stops section numbering
\label{sec:introductie}

A module bundler is one of the most essential technologies in modern web development. Nobody makes webapps with plain HTML, CSS and JS anymore. So the choice of which bundler to use is very important.
While Webpack is the most popular there are many others that claim to do the job better than it. Why is it still the king then I wonder. How does Webpack compare to its competitors? Are their claims true? If so, why is Webpack still the most popular? With this paper. 

Many developers give much thought about which framework to use, which back-end and so forth. The module bundler however often doesn't get that much thoutgh. This is because it's less attractive and in many frameworks comes pre-installed. But as your webapp can't run without it, I believe . In this paper, I want to demystify the module bundler and compare the most popular options. Not just in terms of speed and module size, but also its plugins, developer experience, ... .

Hier introduceer je werk. Je hoeft hier nog niet te technisch te gaan.

Je beschrijft zeker:

\begin{itemize}
  \item de probleemstelling en context
  \item de motivatie en relevantie voor het onderzoek
  \item de doelstelling en onderzoeksvraag/-vragen
\end{itemize}

%---------- Stand van zaken ---------------------------------------------------

\section{State-of-the-art}
\label{sec:state-of-the-art}

Much has already been written about module bundlers online. While they provide very useful information, I have yet to find an in-depth analysis of the differences between the major players and what that means for the developer and the output product.

Hier beschrijf je de \emph{state-of-the-art} rondom je gekozen onderzoeksdomein. Dit kan bijvoorbeeld een literatuurstudie zijn. Je mag de titel van deze sectie ook aanpassen (literatuurstudie, stand van zaken, enz.). Zijn er al gelijkaardige onderzoeken gevoerd? Wat concluderen ze? Wat is het verschil met jouw onderzoek? Wat is de relevantie met jouw onderzoek?

Verwijs bij elke introductie van een term of bewering over het domein naar de vakliteratuur, bijvoorbeeld~\autocite{Doll1954}! Denk zeker goed na welke werken je refereert en waarom.

% Voor literatuurverwijzingen zijn er twee belangrijke commando's:
% \autocite{KEY} => (Auteur, jaartal) Gebruik dit als de naam van de auteur
%   geen onderdeel is van de zin.
% \textcite{KEY} => Auteur (jaartal)  Gebruik dit als de auteursnaam wel een
%   functie heeft in de zin (bv. ``Uit onderzoek door Doll & Hill (1954) bleek
%   ...'')

Je mag gerust gebruik maken van subsecties in dit onderdeel.

%---------- Methodologie ------------------------------------------------------
\section{Methodologie}
\label{sec:methodologie}

First I will compare the most popular bundlers theorethicly. How they work and how they may differ.
To compare them in practice will require code-bases of many sizes. Why many sizes, you ask? Because it could be interesting to see how module bundlers deal with smaller projects and larger ones. To see if the advantages of one deminish when the project size in- or decreases. 

I will test each codebase with many bundlers and then observe the differences in numerous categories. One important of which is the developer experience: how easy is it to install? How simple or compley is the config file and the plugins. There are a lot of things to consider  
Hier beschrijf je hoe je van plan bent het onderzoek te voeren. Welke onderzoekstechniek ga je toepassen om elk van je onderzoeksvragen te beantwoorden? Gebruik je hiervoor experimenten, vragenlijsten, simulaties? Je beschrijft ook al welke tools je denkt hiervoor te gebruiken of te ontwikkelen.

%---------- Verwachte resultaten ----------------------------------------------
\section{Verwachte resultaten}
\label{sec:verwachte_resultaten}

Many module bundlers claim speedier builds and smaller bundle sizes. If that's the case for a project of any size remains to be seen. But I do expect to see a difference there. The question will be if those improvements warrant the competitor to be used over Webpack. Developer experience and performance in development mode are defining factors in my opinion as well. Obviously the former will be quite subjective so that's an important factor to note.

%---------- Verwachte conclusies ----------------------------------------------
\section{Verwachte conclusies}
\label{sec:verwachte_conclusies}

I honestly have no idea what I will conclude. That's the most important reason why I want to conduct this research. Webpack has served us well but as the tech world shifts more and more towards applications written with web technologies, the time of newer approaches may have come.

